\chapter*{Abstract} 
Near the kinematical production threshold scattering cross sections gain large logarithmic corrections. This is due to the fact that close to the threshold the radiation of gauge bosons is restricted to be soft. This leads to an imbalance in the cancelling between real and virtual contributions at higher order, leading to large logarithmic contributions. In order to make reliable predictions such contributions must be resummed. In this thesis we make use of an object called a Wilson line to describe the soft radiation. Wilson lines are path-ordered exponentials of the gauge fields. They contain all the kinematical and dynamical information from the gauge sector and are central in taking a geometrical viewpoint of quantum field theory, and in particular quantum chromodynamics. We mainly consider semi-infinite Wilson lines on linear paths, which naturally describes radiation from highly energetic particles. By constructing a special class of Wilson lines, namely Wilson lines on closed paths called Wilson loops, we show how the soft radiation can be fully characterized by a so-called eikonal cross section. From an explicit one-loop calculation of a Wilson loop expectation value we find the universal cusp anomalous dimension. This cusp anomalous dimension is shown to be a central ingredient in evolution equations for Wilson lines, perturbative distributions and subsequently for eikonal cross sections. With the aid of the cusp anomalous dimension we find an exponentiated form of the Drell-Yan cross section up to leading logarithmic order. 