\chapter*{Acknowledgements}
First of all, I want to thank my supervisor, Are Raklev, for practical advice on my master's thesis. The contextual explanations of the underlying physics and its pertinent challenges have provided me with precisely the amount of knowledge necessary to carry out the work in this thesis. I have enjoyed your calm nature and acceptance of my pragmatic approach to writing. I also want to thank Anders Kvellestad for serving as my unofficial co-advisor. You have taken time to have lengthy discussions of possible theoretical approaches and you have given me valuable insights into certain aspects of computing and interpretation of Bayesian statisics. 

I will never forget the friends I gained through my studies. I want to thank you all for enhancing my experience and helped me grow as a human. It has been a pleasure to learn about the inner-workings of Mother Nature beside you. It was the best of times, it was the worst of times. A special thanks to Toshi for awesome long nights working through our bachelor's courseload and long walks after late nights out, to Bennern for emotional support and wisdom to carry out life-changing decisions, to Maria for your friendship and the christmas celebration I got to spend with your family, to Kaspara for your inability to stop talking and being the life-of-the-party, to Une for being the best labpartner I could ever get and unexpectedly arranging a celebration for my birthday last year, to my boy TK for being my ``homie'' for more than 20 years, to Isak and Karianne for your unfiltered speech and so many more. Rest assured, besides a couple examinators, no one will read this thesis. Thus if your name is omitted, no one will ever know.

A final thanks to the Norwegian government for allowing me the opportunity to pursue an education of high quality at no cost and for the computing resources used in this thesis. Now that we are mentioning education, thanks to every single person that has dedicated their life to understanding nature and progressing our collective base of knowledge. Without them, there would be no education to speak of. 
