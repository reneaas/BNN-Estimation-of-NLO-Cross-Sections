% \newpage
%\chapter*{Conclusion}
\addcontentsline{toc}{chapter}{Conclusion}

\section{Closing Remarks and Future Work}
In this chapter we have explored the results of several numerical experiments. We have demonstrated that trained BNNs can significantly reduce the time spent computing NLO predictions. We found that the time spent on such computations were roughly evenly divided for loading the models in from memory and performing the actual forward pass in the neural networks to compute the predictive distribution, its sample mean and sample variance. Moreover we found that the training employed on GPUs with XLA compilation can result in a significant reduction in training time.
We investigated the posterior distribution of the true sampled weights and found them to be multi-modal, consistent with the claims in the literature. We explored the how various hyperparameters used during training of BNNs affected their predictive performance. We found evidence suggesting that a moderate amount of warm-up steps and pretraining positively impacts the performance of the trained models but that an excessive amount exacerbated it. With the vast set of different configurations one can use with BNNs though, these results may not be generalizable to different forms of architectures than the ones we have used and more extensive investigations can be carried out. Finally, we explored the predictive distributions of a trained BNN model and showcased an example of a good predictive distribution and a predictive distribution which missed the mark entirely. We showed that 95 \% of the targets was within $\pm 3 \sigma$ from the sample mean of the predictive distribution for each target. 


Although we have progressed our understanding of the training of BNNs by drawing sampling from the exact posterior with MCMC samplers like HMC and NUTS, there are several question which we have not answered. We propose the following problems to be addressed in the future.
\begin{enumerate}
    \item \textbf{The Convergence Properties of the Markov Chain}. In chapter \ref{chap:mcmc}, we noted that the standard metric to estimate that a Markov chain has converged to its stationary distribution were by use of the scale reduction factor $\hat{R}$. Such convergence statistics is not measured or reported in this thesis. Computing $\hat{R}$ for neural network posteriors is complicated by the non-identifiability of neural networks. Several different neural networks sampled from the same sample space may produce the same predictions, which makes assessing the convergence by studying the elements of the Markov chain itself challenging. Instead we propose the use of $\hat{R}$ computed on the predictions by using the samples in the Markov chain. 
    \item \textbf{Training of BNNs on Larger Datasets}. In this thesis, we have focused on a fairly small dataset of $\sim 16000$ datapoints. At no points have we investigated the added computational expense from computing the potential energy function and its gradient in HMC and NUTS as a function of number of datapoints it needs to be evaluated for. If NLO cross section estimation is to be used with BNNs on larger datasets, the effect it has on the hyperparameters used during training is likely necessary to be redone. The analysis performed in this thesis should at least give information on what hyperparameters that are worth exploring. Training time will likely be much longer but the predictive performance of the trained BNN may become more robust.
    \item \textbf{Sampling Larger Models}. Our analysis has been dealing with a fairly small number of sampled neural networks per model. In each case, we have drawn 1000 neural networks which collectively represented the full BNN model. The number of parameters the models had, spanned from a few hundred to a hundred thousand. A thousand samples drawn from the posterior is a pretty low number owed to the computational expense needed to generate them. The MCMC estimators and the predictive distributions will likely produce better results if more samples are drawn.
    \item \textbf{The Effect of Thinning}. We have operated with a fixed number of sampled skipped between each drawn network. This means that we have performed no analysis of the correlation between successively drawn samples but instead worked with a heuristic that appeared to produce good results. Investigating the \textit{lag}-$l$ autocorrelation successive neural network samples can give valuable information from a practical perspective. Although drawing more samples may be beneficial for the calculation of MCMC estimators, it is not so if the samples are heavily correlated. Both samplers used in this thesis generate successive samples with low correlation in simple cases studied in the literature \cite{nuts,neal2011} but with the complexity of the BNN posterior, this may require a larger amount of thinning. Performing preliminary runs to estimate how correlated succesive samples are will help reduce the necessary amount of sampled needed to be drawn to obtain good statistics from the MCMC estimators. It will also help the practitioner to minimize the amount of thinning and avoid wasting computational resources.
    \item \textbf{The Effect of the Multi-modality of the Posterior on the Predictive Distributions}. Althoug we demonstrated the multi-modality of the posterior distribution of BNNs, we did not investigate its effect on the predictive distribution. After all, it is the predictive distribution we really care about in practice. Due to how computationally expensive it is to sample from the exact posterior, a thorough comparison of sampling from the exact posterior should be compared and contrasted with the use of surrogate distributions for the BNNs parameters. 
    \item \textbf{Other Potential Energy Functions}. In our investigation we have used a Gaussian prior for each neural network parameter and the same likelihood function for each model. It is possible that modifying the potential energy function, either by choosing different priors or modyfing the likelihood function, that the training process can be improved.
\end{enumerate}