\usepackage{tikz}
\usetikzlibrary{positioning}
\usetikzlibrary{calc}
\begin{figure}[h!]
    \centering
    \begin{tikzpicture}[>=stealth',
                        shorten >=1pt,
                        auto,
                        node distance=3cm,
                        thick,
                        main node/.style={circle,
                                          draw,
                                          font=\sffamily\Large\bfseries
                                         }
                       ]

        \node[main node] (A) {a};
        \node[main node] (B) [below right of=A] {b};
        \node[main node] (C) [below left of=A] {c};
        \node[main node] (D) [below left of=C] {d};
        \node[main node] (E) [below left of=B] {e};
        \node[main node] (F) [below right of=B] {f};
        
        
       \path[every node/.style={font=\sffamily\small}]
           (A) edge node [below] {} (C)
           (B) edge node [above] {} (A)
           (C) edge node [above] {} (D)
           (B) edge node [below] {} (E)
               edge node [below] {} (F);
    \end{tikzpicture}
    \caption{A simple binary tree}
    \label{fig:neighborhood}
\end{figure}

