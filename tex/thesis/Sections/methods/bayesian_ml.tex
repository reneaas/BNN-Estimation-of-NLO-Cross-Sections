In this chapter we will introduce the notion of \textit{Bayesian machine learning} (Bayesian ML).
We will start from the classical view of ML
and reformulate it in terms of Bayesian concepts. We will only concern ourselves
with so-called supervised ML models used to solve supervised regression tasks
as it is the only class of problems of interest in this thesis.
We will first introduce the core of ML and its constituent ingredients.
From this we transition to Bayes' theorem and a Bayesian framework for ML.
Finally we discuss Bayesian inference.

\section{The Core of Machine Learning}
The basic conceptual framework of a supervised machine learning problem is as follows. 
Assume a dataset $D$ is a sequence of $N$ datapoints $D = \{(x^{(i)}, y^{(i)})\}_{i=1}^N$,
where $x^{(i)} \in \mathbb{R}^p$ is the set of \textit{features} 
and $y^{(i)} \in \mathbb{R}^d$ is the \textit{target}. 
The next ingredient is to assume the targets can be decomposed as
\begin{equation}\label{eq:model_assumption}
	y = f(x) + \delta,
\end{equation}
for some true function $f : \mathbb{R}^p \to \mathbb{R}^d$ (also known as the \textit{ground truth}), where $\delta \in \mathbb{R}^d$ is introduced to account for random noise. 
The objective is to learn $f(x)$ from the dataset. To this end, we choose a \textit{model class} $\hat{f}(x; \theta)$ 
parameterized by a model parameters $\theta \in \mathbb{R}^m$,
combined with a procedure to infer an estimate of the parameters $\hat{\theta}$ such that the model is as close to $f(x)$ as possible. 
Formally, this means choosing a \textit{metric} $\mathcal{L}$ to quantify the error, called a \textit{loss} function 
(or a \textit{cost} function, but we will adopt the former term in line with the terminology used in the TensorFlow framework), 
and minimize it with respect to the parameters of the model to obtain $\hat{\theta}$ 
using an optimization algorithm. 
For brevity, we will denote the output of a model class as $\hat{y}^{(i)} \equiv \hat{f}(x^{(i)};\theta)$.

\begin{comment}
    \subsection{Model Class and Model Complexity}
    In the last section we used the term model class without any proper definition.
    A model class $\hat{f}(x; \theta)$ is a function parameterized with a parameter $\theta \in \mathbb{R}^d$,
    where $d$ are the dimension of the parameter space. The \textit{model complexity} can loosely be defined as
    how many free parameters there are in the model class, i.e what the number $d$ is.
\end{comment}



\subsection{Loss Functions}
For regression problems, two loss functions $\mathcal{L}$ are commonly chosen. The first is the \textit{residual sum of squares} (RSS) given by
\begin{equation}\label{eq:rss}
	\mathcal{L}_\text{RSS} \equiv \text{RSS} = \sum_{i=1}^N \norm{y^{(i)} - \hat{y}^{(i)}}_2^2,
\end{equation}
where $\norm{\cdot}_2$ denotes the $L^2$-norm. The second is the the \textit{mean squared error} (MSE), defined as
\begin{equation}\label{eq:mse}
	\mathcal{L}_\text{MSE}\equiv \text{MSE} = \frac{1}{N}\sum_{i=1}^N\norm{y^{(i)} - \hat{y}^{(i)}}_2^2.
\end{equation}
For optimization purposes, they yield equivalent optimal parameters $\hat{\theta}$, at least in principle.

\subsection{Regularization}
With datasets of limited size, \textit{overfitting} can pose a problem, yielding models that generalize poorly
because they become overly specialized to the dataset on which $\hat{\theta}$ is inferred. 
The implication is that the predicted target on unseen data is unlikely to be correct.
This occurs especially if the model is too complex.   
One strategy to overcome this, is to tack on a regularization term to the loss-function. By \textit{regularization},
we mean an additional term that limits the size of the allowed parameter space.
Hence, regularization imposes a constraint on the optimization problem.


The two most commonly used regularization terms are 
$L^2$-regularization, which adds a term to the loss function as
\begin{equation}\label{eq:loss_l2_reg}
	\mathcal{L} = \mathcal{L}_0 + \frac{\lambda}{2} \norm{\theta}_2^2,
\end{equation}
where $\lambda$ is the so-called \textit{regularization strength}, which is what we call a \textit{hyperparameter},
and $\mathcal{L}_0$ is a loss function with no regularization term.
The second is $L^1$-regularization, which yields a loss 
\begin{equation}\label{eq:loss_l1_reg}
	\mathcal{L} = \mathcal{L}_0 + \frac{\lambda}{2}\norm{\theta}_1.
\end{equation}
The terms \textit{penalize} large values of $\theta$, effectively shrinking the allowed parameter space.
The larger the value of the regularization strength $\lambda$, the smaller the allowed parameter space becomes.


More generally, we can decomposed our full loss function as
\begin{equation}\label{eq:loss_fn}
    \mathcal{L}(x, y, \theta) = \mathcal{L}_0 + R(\lambda_1, \ldots, \lambda_r, \theta),
\end{equation}
where $R(\theta)$ is a linear combination of $L^p$-regularization terms where $\lambda_i$
are the expansion coefficients which are all treated as hyperparameters. $L^p$-regularization terms
is defined by the $L^p$-norm
\begin{equation}
    \norm{x}_p = \left(\abs{x_1}^p + \cdots + \abs{x_m}^p\right)^{1/p}, \quad x \in \mathbb{R}^m.
\end{equation}
In practice, we typically use a single form of $L^p$-regularization but nothing
stops us from constructing complicated regularization terms in theory.
\subsection{Optimization}
Once a model class and loss function is chosen, an \textit{optimizer} or \textit{optimization algorithm} must be chosen.
By this, we mean an algorithm that uses the loss function and the model class, and minimizes the loss
with respect to the model parameters to yield an estimate of $\hat{\theta}$. Regardless
of which optimization algorithm we employ, we seek 
\begin{equation}\label{eq:optimal_param}
    \hat{\theta} = \text{arg min}_\theta \ \mathcal{L}.
\end{equation}
In this thesis, optimization plays a smaller role in the inference of model parameters
than in classical ML because we do not seek a single estimate $\hat{\theta}$ in most Bayesian applications.
We shall nevertheless utilize such algorithms for some parts but for another purpose. One of the most popular
optimizers in the deep learning community is ADAM \cite{ADAM} which we will mainly use
when optimization is needed.



\section{Bayes' theorem}
Our goal is to reformulate ML in terms of Bayesian concepts. The backbone of Bayesian ML
is \textit{Bayes' theorem} \cite{bayes_theorem}. The theorem can be formulated as
\begin{equation}
	p(\theta | D) = \frac{p(D|\theta)p(\theta)}{p(D)},
\end{equation}
where $D$ is observed data and $\theta$ denotes the parameters of the model.
Here $p(\theta)$ is called the \textit{prior} distribution and embodies our prior knowledge of $\theta$ before any new observations are considered. 
$p(D|\theta)$ is called the \textit{likelihood} function and provides the relative probability  
of observing $D$ for a fixed value of $\theta$. It need not be normalized to unity, which is why it only provides relative ``probabilities''.
The \textit{posterior} distribution $p(\theta|D)$ models our belief about $\theta$ after the data $D$ is observed. 
Finally, $p(D)$ is called the \textit{evidence} which we may regard as the normalization constant of the posterior
such that posterior integrates to unity over parameter space.
In the context of Bayesian ML, the evidence will not be an interesting quantity as it will not
turn up as part of any algorithms. Moreover, it is typically intractible for sufficiently large parameter spaces.
It is therefore common to write Bayes' theorem as 
\begin{equation}\label{eq:bayes_theorem}
  p(\theta|D) \propto p(D|\theta)p(\theta),
\end{equation}
which we too shall adopt.

\section{Bayesian Framework for Machine Learning}
The Bayesian framework for ML differs somewhat in approach to its classical counterpart.
We define a model class in the same way as before. Choosing a loss function is substituted with
choosing a likelihood function and a prior.
Minimization of the loss function is replaced with maximization of the likelihood function or the posterior distribution. In fact,
The Bayesian framework introduces several ways to infer an estimate for the optimal model parameters \cite{ml_for_physicists}.
\begin{enumerate}
    \item \textit{Maximum Likelihood Estimation} (MLE): The optimal parameters $\hat{\theta}$ are inferred by
    \begin{equation}\label{eq:mle}
        \hat{\theta} = \text{arg max}_\theta \ p(D|\theta),
    \end{equation}
    meaning we choose $\hat{\theta}$ as the mode of the likelihood function.
    This is equivalent to maximizing the log-likelihood (since log is a monotonic function), i.e.
    \begin{equation}\label{eq:map}
        \hat{\theta} = \text{arg max}_\theta \log p(D|\theta).
    \end{equation}
    \item \textit{Maximum-A-Posteriori} (MAP): This estimate of $\hat{\theta}$ is defined as
    \begin{equation}
        \hat{\theta} = \text{arg max}_\theta \ p(\theta|D),
    \end{equation}
    meaning we choose $\hat{\theta}$ as a mode of the posterior distribution.
    \item \textit{Bayes' estimate}: The estimate of $\hat{\theta}$ is chosen as the expectation of the posterior,
    \begin{equation}
        \hat{\theta} = \mathbb{E}_{p(\theta|D)}[\theta] = \int \dd \theta \ \theta p(\theta|D).
    \end{equation}
\end{enumerate}

The connection between classical and Bayesian ML can be understood from what follows.
First, let us assume that each datapoint $(x^{(i)}, y^{(i)})$ is identically and independently distributed (i.i.d.).
The likelihood function can then generally be written as
\begin{equation}
    P(D|\theta) = \prod_{i=1}^N P(y^{(i)}|x^{(i)}, \theta).
\end{equation}
For regression tasks, the standard choice of likelihood function is the \textit{Gaussian}
\begin{equation}
    p(y|x, \theta) = \exp\left(-\frac{1}{2\sigma^2}\norm{y - \hat{f}(x;\theta)}_2^2\right),
\end{equation}
where $\sigma$ is some hyperparameter typically chosen to be the same for every datapoint $(x,y)$.
For the full dataset, we get
\begin{equation}\label{eq:likelihood_fn}
    p(D|\theta) = \prod_{i=1}^N \exp \left(-\frac{1}{2\sigma^2}\norm{y^{(i)} - \hat{f}(x^{(i)};\theta)}_2^2\right).
\end{equation}
Now, consider the definition of MLE from eq.~\eqref{eq:mle}.
It instructs us to maximize the expression in eq.~\eqref{eq:likelihood_fn}. If we rewrite the likelihood function a bit
\begin{equation}
    p(D|\theta) = \exp \left(-\frac{1}{2\sigma^2}\sum_{i=1}^N\norm{y^{(i)} - \hat{f}(x^{(i)};\theta)}_2^2\right),
\end{equation}
we can observe that maximization of the likelihood function simply amounts to minimization of the RSS
and hence of the MSE, as can be seen by comparison with the expressions in eq.~\eqref{eq:rss} and eq.~\eqref{eq:mse}.

We can go even further, by considering the MAP estimate. Let us introduce a Gaussian prior on the parameters
such that
\begin{equation}\label{eq:gaussian_prior}
    p(\theta) \propto \exp\left(-\frac{\lambda}{2}\norm{\theta}_2^2\right).
\end{equation}
The posterior obtained from Bayes' theorem in eq.~\eqref{eq:bayes_theorem} by combining the prior introduced
in eq.~\eqref{eq:gaussian_prior} and the likelihood function in eq.~\eqref{eq:likelihood_fn} is
\begin{equation}
    p(\theta|D) \propto p(D|\theta)p(\theta) \propto \prod_{i=1}^N \exp \left(-\frac{1}{2\sigma^2}\norm{y^{(i)} - \hat{f}(x^{(i)};\theta)}_2^2\right)\exp\left(-\frac{\lambda}{2}\norm{\theta}_2^2\right),
\end{equation}
which we can rewrite as
\begin{equation}
    p(\theta|D) \propto \exp \left(-\left[\frac{1}{2\sigma^2}\sum_{i=1}^N\norm{y^{(i)} - \hat{f}(x^{(i)};\theta)}_2^2 + \frac{\lambda}{2}\norm{\theta}_2^2\right]\right).
\end{equation}
Maximization of this expression is equivalent to minimization of RSS or MSE with a $L^2$-regularization term tacked on 
which can be seen by comparison with eq.~\eqref{eq:loss_l2_reg}. 
Obviously, we are missing a factor $1/N$ in front of the likelihood
term which can be thought of as baked into the $\sigma$ parameter. 
The natural generalization is that the posterior can be expressed as
\begin{equation}\label{eq:posterior_function_of_loss}
    p(\theta|D) \propto \exp\left(-\mathcal{L}\right),
\end{equation}
for any loss function as in eq.~\eqref{eq:loss_fn}. For a purpose that comes much later when
we discuss Hamiltonian Monte Carlo, we can invert eq.~\eqref{eq:posterior_function_of_loss}
\begin{equation}
    \mathcal{L} = -\log Z - \log p(D|\theta) - \log p(\theta),
\end{equation}
for some appropriate normalization constant $Z$. Assuming that the dataset consists of observations that are i.i.d, we get
\begin{equation}\label{eq:loss_function_of_posterior}
    \mathcal{L} = -\log Z - \sum_{i=1}^N p(y^{(i)}|x^{(i)}, \theta) - \log p(\theta).
\end{equation}
Equation~\eqref{eq:loss_function_of_posterior} will play an important role later on.


\section{Bayesian Inference}
We have seen that there is a straight forward connection between the Bayesian framework
and the classical view of ML by looking at estimators $\hat{\theta}$. 
In regression tasks, however, we are seldom interested in a single estimate of the model parameter.
Instead we seek to obtain the posterior distribution from which we can infer other quantities.
In applications where the model class is sufficiently complex, direct computation
of the posterior is not feasible. Instead, we must settle with an approximate posterior distribution
which we construct using Monte Carlo Markov chains (MCMC) methods. 
The discussion of such methods is allocated to chapter~\ref{chap:mcmc}. For now we assume that there
exists a way to generate samples $\theta \sim p(\theta|D)$. We approximate the posterior
by sampling a set of model parameters $\{\theta^{(1)}, \ldots, \theta^{(n)}\}$ where $\theta^{(t)} \sim p(\theta|D)$, 
yielding an \textit{empirical} posterior distribution. 

We will primarily use the posterior to compute two classes of mathematical objects. 
The first is the \textit{predictive distribution} of a target $y^*$ given
an input $x^*$. The predictive distribution can be expressed as
\begin{equation}\label{eq:predictive_distribution}
    p(y^*|x^*, D) = \int \dd\theta \ p(y^*|x^*, \theta)p(\theta|D).
\end{equation}
Equation~\eqref{eq:predictive_distribution} is generally intractible since we cannot exactly compute the posterior.
The predictive distribution is therefore approximated by generating a set of predictions
using the empirical posterior distribution. That is, we indirectly sample from $p(y^*|x^*, D)$
by computation of $\hat{f}(x^*;\theta^{(t)})$ for $t=1,\ldots,n$. In other words, the empirical predictive distribution
is generated as follows.
\begin{equation}\label{eq:predictive_dist_approx}
    \begin{split}
        \theta^{(t)} & \sim p(\theta|D), \\
        f(x^*;\theta^{(t)}) & \sim p(y^*|x^*, \theta).
    \end{split}
\end{equation}

The second class is expectation values with respect to the posterior distribution, which
for a target function $f(\theta)$ is defined as
\begin{equation}\label{eq:bayesian_expval}
    \mathbb{E}_{p(\theta|D)}[f] = \int \dd\theta \ f(\theta)p(\theta|D).
\end{equation}
An important example of eq.~\eqref{eq:bayesian_expval} is the expectation value
of the predictive distribution, which will be the expectation of the model class 
with respect to the posterior
\begin{equation}
    \hat{y} \equiv \mathbb{E}_{p(\theta|D)}[\hat{f}(x;\theta)] = \int \dd\theta \ \hat{f}(x;\theta)p(\theta|D).
\end{equation}
Equation~\eqref{eq:bayesian_expval} must be approximated since we cannot hope to evaluate the posterior $p(\theta|D)$.
Even if we could, we will be working with sufficiently large parameters spaces such that the integral itself
is intractible in any case. Approximation of expectation values is done using MCMC methods 
which is the subject of the next chapter.