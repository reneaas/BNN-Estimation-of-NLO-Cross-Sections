In this chapter, we will explore the details of Hamiltonian Monte Carlo. It is a Markov chain Monte Carlo method that uses gradient-informed
steps to generate a proposal state for Metropolis correction. This is achieved by usage of Hamiltonian dynamics
which allow gradient-informed exploration by treating the model parameters
as ``coordinates'' of a fictitous physical system, and introducing auxilliary variables representing its momenta. 
The coordinates and momenta are required to obey a particular set of coupled differential equations called Hamilton's equations.
The differential equations cannot in general be solved exactly and are instead simulated. 
The particular kind of numerical method used to achieve this
is called the Leapfrog integrator. At the end of a simulation, a new set of coordinates and momenta 
will be generated,
which is regarded as the proposal state to undergo Metropolis correction.
If accepted, we keep the proposed coordinates as the next parameter in the Markov chain. Otherwise, the initial coordinates
assume this role. The auxilliary momenta is discarded and resampled on each iteration as they play no important role for the actual Markov chain.

We begin by presenting Hamiltonian dynamics and the Leapfrog integrator.
Once established we show how the framework is used to construct an MCMC method.
Next, we will see how to apply the method to Bayesian machine learning models before we finalize the
chapter with a discussion on some limitations of the method.



\section{Hamiltonian Dynamics}\label{sec:hamiltonian_dynamics}
\textit{Hamiltonian dynamics} \cite{classical_mechanics} is a formulation of classical mechanics that allows us to compute the time evolution
of a physical system. The fundamental mathematical object of the theory is the \textit{Hamiltonian} $H$ which governs the time evolution of the \textit{coordinates} $q = (q_1, \ldots, q_d)$ and \textit{momenta} $p = (p_1,\ldots, p_d)$ of the system. 
The $2d$-dimensional space defined by the points $(q, p)$ is called \textit{phase-space}.
The precise relationship is formulated by \textit{Hamilton's equations}
\begin{equation}\label{eq:hamiltons_eqs}
  \dv{q_i}{t} = \pdv{H}{p_i}, \qquad \dv{p_i}{t} = - \pdv{H}{q_i}, \qq{for} i=1,\ldots, d.
\end{equation}
The objective is to use the $2d$ coupled differential equations in eq.~\eqref{eq:hamiltons_eqs} to find $(q(t), p(t))$ given some initial condition $(q(0), p(0))$ where $t$ represents time. A system governed by Hamilton's equations is called a \textit{Hamiltonian system}. For the purpose of constructing a MCMC method, we need not consider the most general theory of Hamiltonian dynamics and we will therefore refrain from doing so. We shall confine our focus to Hamiltonians which can be decomposed as
\begin{equation}\label{eq:hamiltonian}
  H(q, p) = V(q) + K(p),
\end{equation}
where $V$ is the \textit{potential energy} and $K$ is the \textit{kinetic energy} of the system. The particular kind of Hamiltonian in eq.~\eqref{eq:hamiltonian} corresponds to the total energy of the system. A key feature is that this Hamiltonian is \textit{conserved} through time. This observation follows from
\begin{equation}
  \dv{H}{t} = \sum_i\left(\dv{q_i}{t}\pdv{H}{q_i} + \dv{p_i}{t}\pdv{H}{p_i}  \right)
  = \sum_i\left(\pdv{H}{p_i}\pdv{H}{q_i} - \pdv{H}{q_i}\pdv{H}{p_i}  \right) = 0.
\end{equation}
Thus any solution $(q(t), p(t))$ will be confined to a hyperplane defined by the Hamiltonian and the initial condition.

Evolving a Hamiltonian system from some initial point $(q(0), p(0))$ is in general a non-trivial task. Exact solutions can only be computed for simple systems. To arm ourselves with a robust MCMC method, then, we must employ a numerical method to approximate the solutions. 
To this end, there is a class of numerical methods called \textit{symplectic integrators} that take advantage of the underlying geometry enforced by Hamilton's equations which allow accurate solutions over long time periods at a lower computational cost than typical higher-order methods such as fourth-order Runge-Kutta.
The particular symplectic integrator used in HMC is called the \textit{Leapfrog integrator} which we shall discuss next.

\subsection{Leapfrog integration}
The Leapfrog integrator \cite{leapfrog} is used in HMC to integrate eq.~\eqref{eq:hamiltons_eqs} to generate new proposal states. 
First assume that we \textit{discretize} the time-coordinate $t$ into 
discrete time coordinates defined by an initial time $t_0$ and a \textit{step size} $\epsilon$ which defines the distance between each time coordinate.
The $k$-th time coordinate can be generated by 
\begin{equation}
  t_k = t_0 + k\epsilon.
\end{equation}
To please mathematicians, we introduce functions $\hat{q}$ and $\hat{p}$ to represent the discretized approximations to the exact solution $(q(t), p(t))$. From an initial point $(q(t_0), p(t_0))$, we simulate the system to obtain approximate values of the exact solution at discrete times $t_1, \ldots, t_n$. 

Consider a single Leapfrog step from a point $(\hat{q}(t), \hat{p}(t))$. 
Its approximation to $(q(t+\epsilon), q(t + \epsilon))$ is then computed as formulated in algorithm~\ref{algo:leapfrog}.
\begin{figure}[H]
	\begin{algorithm}[H]
		\caption{Leapfrog Integration}\label{algo:leapfrog}
		\begin{algorithmic}
      \Function{{\tt Leapfrog}}{$V, q, p, \epsilon$} 
      \For{$i=1,\ldots, d$}
        \State $p_i' \gets p_i - \displaystyle{\frac{\epsilon}{2}\pdv{V(q)}{q_i}}$
        \State $q_i' \gets q_i + \displaystyle{\frac{\epsilon}{m_i}p_i'}$
        \State $p_i' \gets p_i' - \displaystyle{\frac{\epsilon}{2}\pdv{V(q')}{q_i}}$
      \EndFor
      \State \Return $(q', p')$
      \EndFunction
		\end{algorithmic}
	\end{algorithm}
\end{figure}
\noindent Note the introduction of the \textit{masses} $m_i$. For now they may simply be regarded as some constants belonging to the Hamiltonian system. 
When used in HMC, it is common to set all masses $m_i = 1$ from which we can formulate the algorithm in vectorized form, as seen in algorithm~\ref{algo:vec_leapfrog}.
\begin{figure}[H]
	\begin{algorithm}[H]
		\caption{Vectorized Leapfrog Integration}\label{algo:vec_leapfrog}
		\begin{algorithmic}
      \Function{{\tt VectorizedLeapfrog}}{$V, q, p, \epsilon$} 
      \State $p' \gets p - \displaystyle{\frac{\epsilon}{2}}\nabla_q V(q)$
      \State $q' \gets q + \epsilon p'$
      \State $p' \gets p' - \displaystyle{\frac{\epsilon}{2}}\nabla_q V(q')$
      \State \Return $(q', p')$
      \EndFunction
		\end{algorithmic}
	\end{algorithm}
\end{figure}

\section{Generating a Proposal State}
Our next objective is to understand how we connect an arbitrary target distribution $\pi(\theta)$
to Hamiltonian dynamics. In this section we will weave these together and show how we
generate a new proposal state $\theta'$ which will undergo a Metropolis correction.
The results discussed in this section can be understood as representing the proposal distribution $q(\theta'|\theta)$ used during the Metropolis-Hastings step,
as we summarized in algorithm~\ref{algo:general_metropolis}. 

The fundamental assumption we make is that the target distribution can be expressed in terms of a
canonical distribution over coordinate space
\begin{equation}\label{eq:canonical_coordinate}
  \pi(q) \propto \exp\left\{-V(q)\right\},
\end{equation}
where $q$ represents the model parameters $\theta$. We will stick to this convention to avoid confusion and 
utilize the formulation of Hamiltonian dynamics discussed hitherto. Once we want to apply it in a Bayesian ML context,
we simply replace $q \to \theta$. From eq.~\eqref{eq:canonical_coordinate}, we can find the potential
energy function in terms of the target distribution
\begin{equation}\label{eq:potential_energy}
  V(q) = - \log \pi(q),
\end{equation}
up to a constant. Hence once the target distribution is known,
we use eq.~\eqref{eq:potential_energy} to obtain the potential energy of the system.

We now turn to the problem of constructing the Hamiltonian so we can utilize Hamilton's equations.
To achieve this, we must introduce \textit{auxilliary} momenta $p$ so we can define a kinetic energy function
and evolve the system through what we may regard as \textit{fictitious time} $t$. 
The momenta are sampled from some distribution of our own choice. We can proceed in the same way as we did
with the potential energy function and express the momentum distribution in terms of a canonical distribution
over momentum space
\begin{equation}
  \pi(p) \propto \exp\left\{-K(p)\right\},
\end{equation}
such that
\begin{equation}
  K(p) = -\log \pi (p),
\end{equation}
up to a constant. The commonly chosen expression for kinetic energy is 
the one found in classical physics
\begin{equation}\label{eq:K_classical_physics}
  K(p) = \sum_{i=1}^d \frac{p_i^2}{2m_i},
\end{equation}
from which the canonical distribution is inferred to be
\begin{equation}\label{eq:canonical_p}
  \pi(p) \propto \exp\left\{-\sum_{i=1}^d \frac{p_i^2}{2m_i}\right\} = \prod_{i=1}^d \exp\left\{-\frac{p_i^2}{2m_i}\right\}.
\end{equation}
Hence, with the kinetic energy from eq.~\eqref{eq:K_classical_physics}, we 
sample each momentum independently from a Gaussian distribution with zero mean and variance $\sigma_i^2 = m_i$.

Now that we understand how we specify the potential energy for a given target distribution and
the kinetic energy of the auxilliary momenta, we can formulate the full canonical distribution over phase-space as
\begin{equation}\label{eq:full_canonical}
  \pi(q, p) = \pi(q)\pi(p) \propto \exp\left\{-V(q)\right\}\exp\left\{-K(p)\right\} = \exp\left\{-H(q, p)\right\}.
\end{equation}
We are naturally just interested in generating a new coordinate $q'$. Using Hamilton's equations
with the Hamiltonian implied by eq.~\eqref{eq:full_canonical}, we can simulate the fictitious Hamiltonian
system using Hamilton's equation in eq.~\eqref{eq:hamiltons_eqs} to generate a new state $(q', p')$. The proposal state
is then obtained by the projection map $(q', p') \mapsto q'$. 

As stated in the beginning of this section, we may regard the details outlined here as an elaborate explanation of the proposal distribution $q(\theta'|\theta)$. The final keypoint to consider is how we can make it symmetric so
that we only need to evaluate $\pi(q', p')$ at the Metropolis step using eq.~\eqref{eq:symmetric_acceptance_prob}. It can be shown that we only need two additional steps. We must randomly choose to sample forwards or backwards in time. The second step is the negate the momenta at the end of the generation of the state, $p \mapsto -p$. The acceptance probability can then be computed as
\begin{equation}
  a = \min \left(1, \frac{\pi(q', p')}{\pi(q, p)}\right) = \min \left(1, \exp\left\{-\left[H(q',p') - H(q, p)\right]\right\}\right).
\end{equation}
But this should always be evaluated to $a = 1$ if $H$ is indeed conserved. But the catch is that the dynamics is only approximated using the Leapfrog integrator. The best the integrator can do is conserve $H$ on average, with its value oscillating about the initial value.

Before we summarize the algorithm in a neat manner, we shall briefly outline it conceptually. 
\begin{enumerate}
  \item Given an initial state $q$, we randomly sample the auxilliary momenta $p$ from the distribution in eq.~\eqref{eq:canonical_p}
  to generate an initial condition $(q, p)$ to use with Hamilton's equations.
  \item We randomly choose to simulate the system forwards or backwards in time by sampling a variable $v \sim \text{Uniform}(\{-1, 1\})$
  from which the step size is set as $v\epsilon$. Forwards in time is represented by $v = 1$ and backwards in time is represented by $v = -1$. 
  \item Perform $L$ Leapfrog steps using algorithm~\ref{algo:leapfrog} for a total trajectory length of $\epsilon L$ to produce
  a proposal point $(q', p')$.
  \item Perform a Metropolis-Hastings correction on the proposal state to accept or reject it.
  \item Project the phase-space point onto coordinate space and return $q'$ if accepted, or $q$ if rejected, in the previous step.
\end{enumerate}
This essentially summarizes the practical steps of HMC. The introduction of randomly simulating
forwards and backwards in time is to ensure that the algorithm is reversible and obeys the detailed balance condition discussed in chapter~\ref{chap:mcmc}.
To please mathematicians once more, we must really reverse the sign of the final momenta as well, but since we shall use a Gaussian distribution, changing the sign
of the momenta makes no difference to the value of the kinetic energy. 
To generate a Markov chain by this procedure, we simply feed the returned coordinate state back in to the machinery 
and reiterate.
The HMC scheme is summarized in algorithm \ref{algo:hmc}.
\begin{figure}[H]
	\begin{algorithm}[H]
		\caption{Hamiltonian Monte Carlo}\label{algo:hmc}
		\begin{algorithmic}
      \Function{{\tt HMCstep}}{$q, H, L, \epsilon$}
      \State Sample $p \sim \mathcal{N}(0, \text{diag}(m_1, \ldots, m_d))$ \Comment{Sample auxilliary momenta}
      \State Sample $v \sim \text{Uniform}(\{-1, 1\})$. \Comment{Randomly choose direction in time.}
      \State $(q', p') \leftarrow (q, p)$    \Comment{Initialize the initial state.}
      \For{$l = 1, ..., L$} \Comment{Simulate Hamiltonian system for $L$ Leapfrog steps.}
        \State $(q', p') \leftarrow$ {\tt Leapfrog}$(q', p', v\epsilon)$ 
      \EndFor
      \State $a = \min \left(1, \exp\left\{-\left[H(q',p') - H(q, p)\right]\right\}\right)$ \Comment{Compute acceptance probability}
      \State Sample $u \sim U(0,1)$ \Comment{Uniform distribution on $(0,1)$.}
      \If {$a \geq u$} \Comment{Perform Metropolis-Hastings correction}
        \State $q \gets q'$ \Comment{Accept proposed state.}
      \Else
        \State $q \gets q$ \Comment{Reject proposed state.}
      \EndIf
      \State \Return $q$
      \EndFunction
		\end{algorithmic}
	\end{algorithm}
\end{figure}

\section{The Potential Energy Function in Bayesian Machine Learning Applications}
We seek to use HMC in a Bayesian ML application. It is therefore important to
discuss a general way to construct the potential energy function in such applications.
First, recall from chapter \ref{chap:bayesian_ml} in eq.~\eqref{eq:posterior_function_of_loss} that the posterior could in general be written as
\begin{equation}
  p(\theta|D) \propto \exp\left\{-\mathcal{L}(\theta)\right\},
\end{equation}
where $\mathcal{L}$ was some loss function in the classical ML sense.
However, we do not need the evidence term and simply sample from the target distribution $\pi(\theta) = p(D|\theta)p(\theta)$ instead.
Comparison with eq.~\eqref{eq:canonical_coordinate} makes it clear that the potential energy
function simply is $\mathcal{L}$. Combining this with eq.~\eqref{eq:loss_function_of_posterior}, lets us conclude that the general expression
for the potential energy is
\begin{equation}
  \mathcal{L} = - \log p(D|\theta) - \log p(\theta),
\end{equation}
up to a constant.
If we assume all $N$ datapoints are i.i.d. we can recast it as eq.~\eqref{eq:loss_function_of_posterior}, that is
\begin{equation}\label{eq:potential_energy_bayesian}
  \mathcal{L} = -\sum_{i=1}^N \log p(y^{(i)}|x^{(i)}, \theta) - \log p(\theta).
\end{equation}

\section{Limitations of Hamiltonian Monte Carlo}
Although HMC is effective at exploring the state space we wish to sample from, it suffers from the need to hand-tune
the trajectory length $\epsilon L$. Poor choices of $\epsilon$ and $L$ can lead to poor results.
On one hand, if the trajectory length is too short, exploration of the state space will be limited 
which makes HMC behave like a random-walk. Suppose we fix the trajectory length to a finite,
but sufficiently large value.
If the step size $\epsilon$ is too large,
it can lead to instabilities in the leapfrog integrator, while if its chosen to be too small, it will perform far too many
iterations to make the algorithm worthwhile. 
Tuning these parameters requires preliminary runs for the problem at hand and analysis of so-called \textit{trace statistics}, 
which essentially measures the quality of the generated Markov chain. 

In the next chapter, we will look at algorithms that adaptively sets the trajectory length of HMC,
namely the No-U-Turn sampler combined with dual-averaging of the step size, which allows us to overcome these
limitations and more effectively sample from the target distribution without the need for hand-tuning
and analysis of trace statistics, or reliance on heuristics.

