In this chapter, we will study the details of the Hamiltonian Monte Carlo (HMC) method.
It is a Markov Chain Monte Carlo (MCMC) sampling technique that merges Gibbs sampling, 
Hamiltonian dynamics with a final Metropolis-Hastings update.
It avoids random walk behaviour with a systematical exploration of the state space 
and generates successive samples with smaller correlation using gradient-informed exploration. 
We will begin with a survey of Lagrangian and Hamiltonian dynamics followed by a description of
the \textit{Leapfrog} integrator which is used to simulate the Hamiltonian systems. Once these are established, we will summarize the 
HMC method in a generic manner - applicable to any continuous distribution. Moreover, we will discuss important properties like conservation of the Hamiltonian
and local phase space volume.


\section{Gradient-Informed Exploration}
HMC uses gradient-informed exploration to efficiently explore the typical set.
This is achieved by formation of a vector field that is tangent to the typical set when inside it. Recall that the typical set does \textit{not} coincide with the modes of the target
distribution $\pi(\theta)$. Therefore, the gradient of the target distribution alone is of limited
value because it would move the Markov chain outside of the typical set. HMC overcomes this
problem with an introduction of \textit{auxilliary momentum} variables inspired by analytical mechanics. In much the same way that an planet in orbit is submerged in a gravitational field pointing towards the center of gravity, its trajectory is not aligned with the gravitational field.
Instead, its motion is about the center of gravity which is facilited by the momentum of the planet. In this analogy, the gravitational field plays the role of the gradient of the target distribution and its orbit play the role of the typical set.

\section{Canonical Distribution}


