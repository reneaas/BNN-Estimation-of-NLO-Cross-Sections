In this chapter, we shall discuss the need for application of machine learning models in physics problems
and why this is a necessary solution.

 
\section{Computation of Beyond the Standard Model Cross Sections}
The Standard Model of particle physics (SM) is a successful fundamental theory that describes fundamental particles and their interactions.
Despite its success, however, it has a few limitations on its own which has led physicists to propose extentions to the model to explain physics that SM cannot. One such family of extensions is called \textit{supersymmetry}. 

\section{Tmp}
\begin{itemize}
    \item $n_i$: Målte events (kollisjoner) som oppfyller et sett (kalt signalregion) med kriterier (``cuts'').
    \item $b_i$: Bakgrunnen. Estimert SM bidrag for samme signal region.
    \item $s_i$: BSM estimert bidrag for signalregionen med et sett med parameterverdier for en ny BSM modell (i.e SUSY). Den er regnet ut ved 
    \begin{equation}
        s_i = \sigma \epsilon_i A_i \mathcal{L},
    \end{equation}
    der $\sigma$ er tverrsnittet som måler sannsynligheten for at en ``ny'' prosess skjer, $\epsilon_i$ er detektor effektivitet, $A_i$ er akseptans 
    og $\mathcal{L}$ er integrert luminositet over data brukt i søket. 
    \item Statistisk analyse gjøres med å regne ut Poisson likelihood 
    \begin{equation}
        \mathcal{L}(s, b, n) = \frac{e^{-(s + b)}(s + b)^n}{n!}.
    \end{equation} 
    og en test statistikk
    \begin{equation}
        q = -2\ln \frac{\mathcal{L}(s, b, n)}{\mathcal{L}(s=0, b, n)}.
    \end{equation}
\end{itemize}


